%%%%%%%%%%%%%%%%%%%%%%%%%%%%%%%%%%%%%%%%%
% Short Sectioned Assignment
% LaTeX Template
% Version 1.0 (5/5/12)
%
% This template has been downloaded from:
% http://www.LaTeXTemplates.com
%
% Original author:
% Frits Wenneker (http://www.howtotex.com)
%
% License:
% CC BY-NC-SA 3.0 (http://creativecommons.org/licenses/by-nc-sa/3.0/)
%
%%%%%%%%%%%%%%%%%%%%%%%%%%%%%%%%%%%%%%%%%

%----------------------------------------------------------------------------------------
%	PACKAGES AND OTHER DOCUMENT CONFIGURATIONS
%----------------------------------------------------------------------------------------

\documentclass[paper=a4, fontsize=11pt]{scrartcl} % A4 paper and 11pt font size

\usepackage[T1]{fontenc} % Use 8-bit encoding that has 256 glyphs
%\usepackage{fourier} % Use the Adobe Utopia font for the document - comment this line to return to the LaTeX default
\usepackage[english]{babel} % English language/hyphenation
\usepackage{amsmath,amsfonts,amsthm} % Math packages
\usepackage[margin=2cm]{geometry}
\usepackage{lipsum} % Used for inserting dummy 'Lorem ipsum' text into the template
\usepackage{fancyvrb}
\usepackage{enumitem}

\usepackage{sectsty} % Allows customizing section commands
\allsectionsfont{\centering \normalfont\scshape} % Make all sections centered, the default font and small caps

\usepackage{fancyhdr} % Custom headers and footers
\pagestyle{fancyplain} % Makes all pages in the document conform to the custom headers and footers
\fancyhead{} % No page header - if you want one, create it in the same way as the footers below
\fancyfoot[L]{} % Empty left footer
\fancyfoot[C]{} % Empty center footer
\fancyfoot[R]{\thepage} % Page numbering for right footer
\renewcommand{\headrulewidth}{0pt} % Remove header underlines
\renewcommand{\footrulewidth}{0pt} % Remove footer underlines
\setlength{\headheight}{13.6pt} % Customize the height of the header

\numberwithin{equation}{section} % Number equation*s within sections (i.e. 1.1, 1.2, 2.1, 2.2 instead of 1, 2, 3, 4)
\numberwithin{figure}{section} % Number figures within sections (i.e. 1.1, 1.2, 2.1, 2.2 instead of 1, 2, 3, 4)
\numberwithin{table}{section} % Number tables within sections (i.e. 1.1, 1.2, 2.1, 2.2 instead of 1, 2, 3, 4)

%\setlength\parindent{0pt} % Removes all indentation from paragraphs - comment this line for an assignment with lots of text

%----------------------------------------------------------------------------------------
%	TITLE SECTION
%----------------------------------------------------------------------------------------

\newcommand{\horrule}[1]{\rule{\linewidth}{#1}} % Create horizontal rule command with 1 argument of height

\title{	
\normalfont \normalsize 
\textsc{university, school or department name} \\ [25pt] % Your university, school and/or department name(s)
\horrule{0.5pt} \\[0.4cm] % Thin top horizontal rule
\huge Assignment Title \\ % The assignment title
\horrule{2pt} \\[0.5cm] % Thick bottom horizontal rule
}

\author{John Smith} % Your name

\date{\normalsize\today} % Today's date or a custom date

\begin{document}

\section{Domain}
Our ontology wishes to explore the domain of western dress codes and the way in which they govern outfits. Dress codes are written, and more often, unwritten sets of rules that regulate the types of clothing and the specifications of the clothing to be worn to different occasions and events. A classification of these codes is normally made for varying levels of formality of the occasion and times of day the event is held. 

Western dress codes, albeit a by-product of Western Culture (which in itself can be perceived as very diverse), will be assumed to relate to the rather sweeping notion of western dress that permeates much of the world today. 

As previously mentioned dress codes encompass many unwritten rules. These unwritten rules result from the influences on dress codes by societal norms of the respective time period. We will assume that the dress codes being codified in our axioms are more or less pertinent to present day but acknowledge the chance for their possible antiquity. Some notable  instances of arguable rules of dress include the distinction between formal and semi-formal wear and the level of formality of a dress and skirt's hem length. 

To exercise the effectiveness of our ontology and to accommodate certain use-cases, our ontology will include a brief selection of occasion instances which will be designated specifically one formality. How certain events and occasions are perceived for their formality is greatly influenced by the cultural norms of the persons hosting the event (which deems worthy an ontology in itself). Therefore, event instances included in our ontology will be extremely general and absent of any cultural specificity.

A distinction must be made between a dress code and what is fashionable. We acknowledge that elements of fashion of any time period do affect a dress code. However, we will largely ignore the elements of an item of clothing that may or may not be fashionable present day. Therefore we will only address rules of dress that go as far as what can be said has been a fashion constant of the past several decades e.g. A dark colored/neutral colored men's suit and it's formality dependent variations have more or less been the norm for men for events ranging from informal to formal for the past several decades.

\section{Motivating Scenarios}
The motivating scenario for the development of the ecommerce outfit assembly ontology is to provide a software agent the knowledge required to assemble acceptable outfits for different events and occasions from clothing available through online product catalogues. Via the ontology a software agent could discern between clothing items in a product catalogue on characteristics of formality and targeted gender. The aggregated product information can subsequently be interpreted to build product sets that form complete and logical outfit selections that match the requirements of the user of the software agent.

%commonly understood clothing terms when extracting and structuring product information obtained when parsing the text of product descriptions in their respective catalogues. 
%The ontology can be used to understand the principles of building logical outfits that conform to western dress codes for each gender.

The ontology can be extended with other ecommerce ontologies to allow a software agent to pair product catalogue selections with details of product availability (in size), costs and the proximity of origin of shipment of the clothing articles. This would be done for the purposes of building outfit selections that match the customers requirements in size, style, fastest and/or cheapest shipping options, all within his or her specified budget.
\subsection{Use Case 1}
A female person receives an invite to a charity ball dinner and would like to search her favorite ecommerce clothing websites to know what are her clothing options available for purchase that she can wear in attendance to the function.

\subsection{Use Case 2}
A male diplomat is attending a state dinner and would like to know what are his clothing options and what is the nearest clothing store that sells the required garments to complete the outfit.

\subsection{Use Case 3}
A female person has a wardrobe of dresses and skirts she can wear to her first day at work in the downtown bank tower. She is most likely to be introduced to a number of her coworkers and attend a meeting. She wants to know which dresses or skirts are acceptable options that can be included in a business appropriate outfit for the occasion. 

\subsection{Use Case 4}
The same woman from Use Case 3 realizes that her current wardrobe offers a limited number of clothing options to be worn to work. She fears she won't get through the week without having to wear some outfits to work a second time. She searches her favorite online retailers for new garments that can be purchased to fill the gap that match her requirements.


\section{Informal Competency Questions}
There is a consistent pattern to the types of questions of which the use cases present and the ontology would like to resolve.

\begin{enumerate}
	\item What is the implied formality of a specified occasion and what type of clothing conform to the rules of the dress code associated with the respective formality of the event and the gender of the person?
		\begin{enumerate}[label*=\arabic*.]
			\item Is an evening gown suitable to a casual dinner party in one is a woman?
			\item If one is a man, is a tuxedo appropriate when attending the wedding of your best business client's son?
			\item Is a business suit appropriate when attending a state dinner if a man?
		\end{enumerate}
	\item Given a selection of clothing, are there rules on how the items of clothing must be paired together governed by the dress code?
		\begin{enumerate}[label*=\arabic*.]
			\item Can the articles of a men's semi-formal suit consist of pieces sewn from different fabrics?
			\item Can both a a dress and a blouse considered business attire on their own be worn together in the same informal outfit?
		\end{enumerate}
	\item Given a set of options of clothing by type, what are some unique restrictions on specific types of clothing that would govern their acceptance at a formal occasion, a semiformal occasion or an informal occasion?
	\begin{enumerate}[label*=\arabic*.]
		\item What are the hem lengths acceptable for a woman's dress when attending a formal function.
		\item What are the hem lengths acceptable for a woman's skirt when worn as a skirt suit when going to work in a traditional office.
		\item What are the color options for a man's suit at an informal occasion.
	\end{enumerate}
\end{enumerate}

Albeit not directly stated in the use cases, there are a number of questions that can be extended from those listed above.
\begin{enumerate}
	\item What is the absolute minimum required types of clothing to complete any outfit?
	\item What parts of the body must be covered to consider an outfit complete?
	\item Which types of clothes cover which parts of the body?
	\item How and in what sequence are certain types of clothes worn when worn together?
\end{enumerate}

\section{Signature}
\subsection{Classes}
\begin{itemize}
	\item DomainConcepts
\end{itemize}
\subsection{Relations}
\begin{enumerate}
	\item[dyed]
\end{enumerate}
\section{Formalized Competency Questions}
\begin{enumerate}

	\item Is an evening gown part of an outfit that is appropriate for an opening night theatre performance if one is a female?
	\begin{equation*}
		\begin{split}
		\forall w \forall x \forall y \forall z \; eveningGown(w) \land theatreOpeningNight(x) \land female(y) \land outfit(z) \\
		\land garment\_of(w,z) \supset has\_genderDes(w,z) \land eventAppropriate(z,y)
		\end{split}
	\end{equation*}
	%Prover9
	\begin{Verbatim}[frame=lines,gobble=2,numbers=left]
		(all w all x all y all z 
		(eveningGown(w) & theatreOpeningNight(x) & female(y) & garment_of(w,z))
		->
		(has_genderDes(x,y) & is_eventAppropriate(z,x))).
	\end{Verbatim}

	%- - - - - - - - - - - - - - - - - - - - - - - - - - - 
	
	\item Does a dress and shoes alone complete a business outfit?
%	\begin{equation*}
%		\begin{split}
%		\forall x \forall y \forall z \; dress(x) \land shoes(y) \land outfit(z) \land garment\_of(x,z) \land garment\_of(y,z) \supset \\ 
%		outfit(z)
%		\end{split}
%	\end{equation*}
%	%Prover9
%	\begin{Verbatim}[frame=lines,gobble=2,numbers=left]
%		(all x all y all z
%		(dress(x) & shoes(y) & outfit(z) & garment_of(x,z) & garment_of(y,z))
%		->
%		(businessOutfit(z)).
%	\end{Verbatim}
%
	%- - - - - - - - - - - - - - - - - - - - - - - - - - - 
	
	\item Does a jacket, pants and shoes comprise a complete mens outfit?
	\begin{equation*}
		\begin{split}
		\forall x \forall y \forall z \; dress(x) \land pants(y) \land outfit(z) \land garment\_of(x,z) \land garment\_of(y,z) \supset \\ 
		outfit(z)
		\end{split}
	\end{equation*}
	%Prover9
	\begin{Verbatim}[frame=lines,gobble=2,numbers=left]
		(all x all y all z
		(dress(x) & shoes(y) & outfit(z) & garment_of(x,z) & garment_of(y,z))
		->
		(businessOutfit(z)).
	\end{Verbatim}

	%- - - - - - - - - - - - - - - - - - - - - - - - - - - 
	
	\item Is a three piece suit complete without a vest?
	\begin{equation*}
		\begin{split}
		\forall x \forall y \forall z \; vest(x) \land \neg component\_of(x,y) \supset threePieceSuit(y)
		\end{split}
	\end{equation*}
	%Prover9
	\begin{Verbatim}[frame=lines,gobble=2,numbers=left]
		(all x all y
		(vest(x) & -component_of(x,y))
		->
		(threePieceSuit(y)).
	\end{Verbatim}

	%- - - - - - - - - - - - - - - - - - - - - - - - - - - 
	
	\item Can a man wear a men's suit without a tie to a business meeting?
	
	%- - - - - - - - - - - - - - - - - - - - - - - - - - - 
	
	\item Must a man where men's neck apparel to a semi-formal event?

	%- - - - - - - - - - - - - - - - - - - - - - - - - - - 
	
	\item Is a red colored suit suitable to worn to a charity ball?

	%- - - - - - - - - - - - - - - - - - - - - - - - - - - 
		
	\item Is a green ballroom gown suitable for a state dinner?
	\begin{equation*}
		\begin{split}
		\forall x \forall y \forall z \; ballroomGown(x) \land green(c) \land dyed(x,c) \supset \\ 
		suitable(y,z)
		\end{split}
	\end{equation*}
	%Prover9
	\begin{Verbatim}[frame=lines,gobble=2,numbers=left]
		(all x all y all z
		(ballroomGown(x) & outfit(y) & dinnerParty(z) & garment_of(x,y)
		->
		(suitable(y,z)).
	\end{Verbatim}
	
	%- - - - - - - - - - - - - - - - - - - - - - - - - - - 
	
	\item Is an evening gown part of an outfit suitable for a dinner party?
	\begin{equation*}
		\begin{split}
		\forall x \forall y \forall z \forall c \; eveningGown(x) \land dinnerParty(z) \land garment\_of(x,y) \supset \\
		suitable(y,z)
		\end{split}
	\end{equation*}
	%Prover9
	\begin{Verbatim}[frame=lines,gobble=2,numbers=left]
		(all x all y all z
		(eveningGown(x) & outfit(y) & dinnerParty(z) & garment_of(x,y)
		->
		(suitable(y,z)).
	\end{Verbatim}
	
	%- - - - - - - - - - - - - - - - - - - - - - - - - - - 	
	
	\item Can a woman where a dress above the knee hemline to a semi-formal event?

	%- - - - - - - - - - - - - - - - - - - - - - - - - - - 

	\item Can an evening gown be included in a dress suit?

	%- - - - - - - - - - - - - - - - - - - - - - - - - - -

	\item Can a skirt suit include a skirt with a mid thigh hemline to a business meeting?

	%- - - - - - - - - - - - - - - - - - - - - - - - - - -
	
\end{enumerate}
\clearpage

\section{Axioms}
\subsection{Subclass Axioms}
\begin{enumerate}

	\item A dress is a type of garment
	\begin{equation*}
		\forall x \; dress(x) \supset garment(x)
	\end{equation*}
	%Prover9
	\begin{Verbatim}[frame=lines,gobble=2,numbers=left]
		(all x (dress(x)) -> garment(x))
	\end{Verbatim}
	
	%----------------------------------------

	\item A evening gown is a type of dress
	\begin{equation*}
		\forall x \; eveningGown(x) \supset garment(x)
	\end{equation*}
	%Prover9
	\begin{Verbatim}[frame=lines,gobble=2,numbers=left]
		(all x (eveningGown(x)) -> dress(x))
	\end{Verbatim}

	%----------------------------------------

%Subclass 3
\item Legs, feet and torso are body segments
\begin{equation*}
	\forall x \; legs(x) \lor legs(y) \lor feet(x) \supset bodySegment(x)
\end{equation*}

\end{enumerate}
\clearpage

%\subsection{Disjointness Axioms}
%\begin{enumerate}
%	\item
%\end{enumerate}
%\clearpage

\subsection{Sort Constraints}
\begin{enumerate}
	
	%BEGIN- - - - - - - - - - - - - - - - - - - - - - - - - - - 
	
\item Garment are dyed colors
\begin{equation*}
	\forall x \forall y \; dyed(x,y) \supset garment(x) \land color(y)
\end{equation*}
	
	%- - - - - - - - - - - - - - - - - - - - - - - - - - - 

\item Events suggest dress codes
\begin{equation*}
	\forall x \forall y \; suggest(x,y) \supset event(x) \land dressCode(y)
\end{equation*}
\begin{Verbatim}[gobble=2, numbers=left]
	(all x all y (suggest(x,y))->(event(x) & dressCode(y))).
\end{Verbatim}
	
	%- - - - - - - - - - - - - - - - - - - - - - - - - - - 

\item Outfits conform to dress codes
\begin{equation*}
	\forall x \forall y \; conforms(x,y) \supset outfit(x) \land dressCode(y)
\end{equation*}
\begin{Verbatim}[gobble=2, numbers=left]
	(all x all y (conforms(x,y))->(outfit(x) & dressCode(y))).
\end{Verbatim}

	%- - - - - - - - - - - - - - - - - - - - - - - - - - - 

\item Outfits are suitable for events
\begin{equation*}
	\forall x \forall y \; suitable(x,y) \supset outfit(x) \land event(y)
\end{equation*}
\begin{Verbatim}[gobble=2, numbers=left]
	(all x all y (suitable(x,y))->(outfit(x) & event(y))).
\end{Verbatim}
	
	%- - - - - - - - - - - - - - - - - - - - - - - - - - - 

\item Dress codes govern outfits
\begin{equation*}
	\forall x \forall y \; governs(x,y) \supset dressCode(y,x) \land outfit(y)
\end{equation*}
\begin{Verbatim}[gobble=2, numbers=left]
	(all x all y (governs(x,y))->(dressCode(x) & outfit(y))).
\end{Verbatim}
	
	%- - - - - - - - - - - - - - - - - - - - - - - - - - - 

\item Events permit outfits
\begin{equation*}
	\forall x \forall y \; permits(x,y) \supset event(x) \land outfit(y) 
\end{equation*}
\begin{Verbatim}[gobble=2, numbers=left]
	(all x all y (permits(x,y))->(event(x) & outfit(y))).
\end{Verbatim}
	
	%- - - - - - - - - - - - - - - - - - - - - - - - - - - 

\item Garments are a component of outfits
\begin{equation*}
	\forall x \forall y \; component\_of(x,y) \supset garment(x) \land outfit(y)
\end{equation*}
\begin{Verbatim}[gobble=2, numbers=left]
	(all x all y (component_of(x,y))->(garment(x) & outfit(y))).
\end{Verbatim}
	
	%- - - - - - - - - - - - - - - - - - - - - - - - - - - 

\item Outfits include garments
\begin{equation*}
	\forall x \forall y \; include(x,y) \supset garment(x) \land outfit(y)
\end{equation*}
\begin{Verbatim}[gobble=2, numbers=left]
	(all x all y (include(x,y))->(outfit(x) & garment(y))).
\end{Verbatim}

	%- - - - - - - - - - - - - - - - - - - - - - - - - - - 

\item Outfits are worn by genders
\begin{equation*}
	\forall x \forall y \; worn\_by(x,y) \supset outfit(x) \land gender(y)
\end{equation*}
\begin{Verbatim}[gobble=2, numbers=left]
	(all x all y (worn_by(x,y))->(outfit(x) & gender(y))).
\end{Verbatim}

	%- - - - - - - - - - - - - - - - - - - - - - - - - - - 

\item Garments are targeted to genders
\begin{equation*}
	\forall x \forall y \; worn\_by(x,y) \supset outfit(x) \land gender(y)
\end{equation*}
\begin{Verbatim}[gobble=2, numbers=left]
	(all x all y (worn_by(x,y))->(outfit(x) & gender(y))).
\end{Verbatim}

	%- - - - - - - - - - - - - - - - - - - - - - - - - - - 

\item Garments are worn over garments
\begin{equation*}
	\forall x \forall y \; worn\_over(x,y) \supset garment(x) \land garment(y)
\end{equation*}
\begin{Verbatim}[gobble=2, numbers=left]
	(all x all y (worn_over(x,y))->(garment(x) & garment(y))).
\end{Verbatim}

	%- - - - - - - - - - - - - - - - - - - - - - - - - - - 

\item Garments are worn under garments
\begin{equation*}
	\forall x \forall y \; worn\_under(x,y) \supset garment(x) \land garment(y)
\end{equation*}
\begin{Verbatim}[gobble=2, numbers=left]
	(all x all y (worn_over(x,y))->(garment(x) & garment(y))).
\end{Verbatim}

	%- - - - - - - - - - - - - - - - - - - - - - - - - - - 

\item Garments cover body segments
\begin{equation*}
	\forall x \forall y \; covers(x,y) \supset garment(x) \land bodySegment(y)
\end{equation*}
\begin{Verbatim}[gobble=2, numbers=left]
	(all x all y (covers(x,y))->(garment(x) & bodySegment(y))).
\end{Verbatim}

	%- - - - - - - - - - - - - - - - - - - - - - - - - - - 

\item Body segments are covered by garments
\begin{equation*}
	\forall x \forall y \; covered\_by(x,y) \supset bodySegment(x) \land garment(y)
\end{equation*}
\begin{Verbatim}[gobble=2, numbers=left]
	(all x all y (covered_by(x,y))->(bodySegment(x) & garment(y))).
\end{Verbatim}

	%END- - - - - - - - - - - - - - - - - - - - - - - - - - - 

\end{enumerate}
\clearpage

\subsection{Dependence Axioms}
\begin{enumerate}

	%- - - - - - - - - - - - - - - - - - - - - - - - - - - 
	
	\item All outfits include a garment that is worn over the torso and is not an accompaniment garment
		\begin{multline*}
		\forall(x,y) \; (outfit(x) \land torso(y) \supset (\exists z \; garment(z) \\
		\land worn\_over(z,y) \land component\_of(z,x) \land \neg accompanimentGarment(z))) 
		\end{multline*}
	%Prover9
	\begin{Verbatim}[frame=lines,gobble=2,numbers=left]
	 (all x all y 
	 (outfit(x) & torso(y))
	 -> 
	 (exists z (garment(z) & worn_over(z,y) & component_of(z,x) 
	 & -accompanimentGarment(z)))).
	\end{Verbatim}

	%- - - - - - - - - - - - - - - - - - - - - - - - - - - 
	
	\item All outfits include a garment that is worn over the legs and is not an accompaniment garment
		\begin{multline*}
		\forall(x,y) \; (outfit(x) \land legs(y) \supset (\exists z \; garment(z) \\
		\land worn\_over(z,y) \land component\_of(z,x) \land \neg accompanimentGarment(z))) 
		\end{multline*}
	%Prover9
	\begin{Verbatim}[frame=lines,gobble=2,numbers=left]
	 (all x all y 
	 (outfit(x) & legs(y))
	 -> 
	 (exists z (garment(z) & worn_over(z,y) & component_of(z,x) 
	 & -accompanimentGarment(z)))).
	\end{Verbatim}

	%- - - - - - - - - - - - - - - - - - - - - - - - - - - 
	
	\item All outfits include a garment that is worn over the feet and is not an accompaniment garment
		\begin{multline*}
		\forall(x,y) \; (outfit(x) \land feet(y) \supset (\exists z \; garment(z) \\
		\land worn\_over(z,y) \land component\_of(z,x) \land \neg accompanimentGarment(z))) 
		\end{multline*}
	%Prover9
	\begin{Verbatim}[frame=lines,gobble=2,numbers=left]
	 (all x all y 
	 (outfit(x) & feet(y))
	 -> 
	 (exists z (garment(z) & worn_over(z,y) & component_of(z,x) 
	 & -accompanimentGarment(z)))).
	\end{Verbatim}	

	%- - - - - - - - - - - - - - - - - - - - - - - - - - - 
	
	\item All mens suits consist of pants and a jacket of the same color
		\begin{multline*}
			\forall x \; mensSuit(x) \supset \exists y \exists z \exists c \; mensPants(y) \land mensJacket(z) 
			\\ \land color(c) \land dyed(y,c) \land dyed(z,c)
		\end{multline*}
	%Prover9
	\begin{Verbatim}[frame=lines,gobble=2,numbers=left]
	 (all x 
	 (mensSuit(x)
	 -> 
	 (exists y exists z (mensPants(y) & mensJacket(z) & color(c) & dyed(y,c) 
	 & dyed(z,c)))).
	\end{Verbatim}	

	%- - - - - - - - - - - - - - - - - - - - - - - - - - - 
	
	\item All garments are dyed a color
		\begin{equation*}
			\forall x \; garment(x) \supset \exists c \; color(c) \land dyed(x,c) 
		\end{equation*}
	%Prover9
	\begin{Verbatim}[frame=lines,gobble=2,numbers=left]
	 (all x 
	 (garment(x)) 
	 -> 
	 (exists c (color(c) & dyed(x,c)))).
	\end{Verbatim}	
	
\end{enumerate}
\clearpage

%\subsection{Cardinality}

%\subsection{Uniqueness}


\subsection{Definitions}
\begin{enumerate}

	\item Women's garments are garments targeted to females
	\begin{equation*}
		\forall x \forall y \; womensGarment(x) \equiv garment(x) \land female(y) \land targeted\_to(x,y)
	\end{equation*}
	%Prover9
	\begin{Verbatim}[frame=lines,gobble=2,numbers=left]
	 (all x all y 
	 (womensGarment(x) 
	 <-> 
	 (garment(x) & female(y) & targeted_to(x,y)))).
	\end{Verbatim}

	%- - - - - - - - - - - - - - - - - - - - - - - - - - - 

	\item Men's garments are garments targeted to males
	\begin{equation*}
		\forall x \forall y \; mensGarment(x) \equiv garment(x) \land male(y) \land targeted\_to(x,y)
	\end{equation*}
	%Prover9
	\begin{Verbatim}[frame=lines,gobble=2,numbers=left]
	 (all x all y
	 (mensGarment(x) 
	 <-> 
	 (garment(x) & male(y) & targeted_to(x,y)))). 
	\end{Verbatim}

	%- - - - - - - - - - - - - - - - - - - - - - - - - - - 
	
	\item Lower garments are garment worn over the legs 
	\begin{equation*}
		\forall x \forall y \; lowerGarment(x) \equiv garment(x) \land legs(y) \land worn\_over(x,y)
	\end{equation*}
	%Prover9
	\begin{Verbatim}[frame=lines,gobble=2,numbers=left]
	 (all x all y
	 (lowerGarment(x) 
	 <-> 
	 (garment(x) & legs(y) & worn_over(x,y)))). 
	\end{Verbatim}

	%- - - - - - - - - - - - - - - - - - - - - - - - - - - 
	
	\item Upper garments are garments worn over the torso 
	\begin{equation*}
		\forall x \forall y \; upperGarment(x) \equiv garment(x) \land torso(y) \land worn\_over(x,y)
	\end{equation*}
	%Prover9
	\begin{Verbatim}[frame=lines,gobble=2,numbers=left]
	 (all x all y
	 (upperGarment(x) 
	 <-> 
	 (garment(x) & torso(y) & worn_over(x,y)))). 
	\end{Verbatim}

	%- - - - - - - - - - - - - - - - - - - - - - - - - - - 
	
	\item Footwear are garments worn over the feet 
	\begin{equation*}
		\forall x \forall y \; footwear(x) \equiv garment(x) \land feet(y) \land worn\_over(x,y)
	\end{equation*}
	%Prover9
	\begin{Verbatim}[frame=lines,gobble=2,numbers=left]
	 (all x all y
	 (upperGarment(x) 
	 <-> 
	 (garment(x) & feet(y) & worn_over(x,y)))). 
	\end{Verbatim}

	%- - - - - - - - - - - - - - - - - - - - - - - - - - - 

	\item Men's garments are not women's garments
	\begin{equation*}
		\forall x \; mensGarment(x) \equiv \lnot womensGarment(x)
	\end{equation*}
	%Prover9
	\begin{Verbatim}[frame=lines,gobble=2,numbers=left]
	 (all x 
	 (mensGarment(x) 
	 <->
	 -womensGarments(x))).
	\end{Verbatim}
	
	%- - - - - - - - - - - - - - - - - - - - - - - - - - - 

	\item A formal outfit is an outfit that conform to a formal dress code
	\begin{equation*}
		\forall x \forall y \; formalOutfit(x) \equiv outfit(x) \land formal(y) \land conforms(x,y)
	\end{equation*}
	%Prover9
	\begin{Verbatim}[frame=lines,gobble=2,numbers=left]
	 (all x all y 
	 (formalOutfit(x))
	 <->
	 (outfit(x) & formal(y) & conforms(x,y))).
	\end{Verbatim}

	%- - - - - - - - - - - - - - - - - - - - - - - - - - - 
	
	\item A semiformal outfit is an outfit that conform to a semiformal dress code
	\begin{equation*}
		\forall x \forall y \; semiformalOutfit(x) \equiv outfit(x) \land semiFormal(y) \land conforms(x,y)
	\end{equation*}
	%Prover9
	\begin{Verbatim}[frame=lines,gobble=2,numbers=left]
 (all x all y 
 (semiFormalOutfit(x))
 <->
 (outfit(x) & semiFormal(y) & conforms(x,y))).
	\end{Verbatim}

	%- - - - - - - - - - - - - - - - - - - - - - - - - - - 

	\item An informal outfit is an outfit that conform to an informal dress code
	\begin{equation*}
		\forall x \forall y \; informalOutfit(x) \equiv outfit(x) \land informal(y) \land conforms(x,y)
	\end{equation*}
	%Prover9
	\begin{Verbatim}[frame=lines,gobble=2,numbers=left]
 (all x all y 
 (informalOutfit(x))
 <->
 (outfit(x) & informal(y) & conforms(x,y))).
	\end{Verbatim}	

	%END_SECTION - - - - - - - - - - - - - - - - - - - - - - - - - - - 

\end{enumerate}
\clearpage

\subsection{Properties of Relations}
\subsubsection{Inverse Relation}
	\begin{Verbatim}[frame=lines,gobble=2,numbers=left]
	(all x all y (include(x,y) <-> components_of(y,x))).
	(all x all y (permits(x,y) <-> suitable(y,x))).
	(all x all y (governs(x,y) <-> components_of(y,x))).
	(all x all y (worn_over(x,y) <-> worn_under(y,x))).
	(all x all y (covers(x,y) <-> covered_by(y,x))).
	\end{Verbatim}	

\subsubsection{SubProperty of Relation Chain}
\begin{enumerate}
	
	\item If an event suggest a dress code and that dress code governs a set of outfits than that event permits those outfits.
	\begin{equation*}
		\forall x \forall y \forall z \; suggest(x,y) \land governs(y,z) \supset permits(x,z)
	\end{equation*}
	\begin{Verbatim}[frame=lines,gobble=2,numbers=left]
		(all x all y all z 
		(suggest(x,y) & governs(y,z))
		->
		(permits(x,z))).
	\end{Verbatim}

	%- - - - - - - - - - - - - - - - - - - - - - - - - - - 
	
	\item All components of an outfit that is worn by a gender are targeted to that gender 
	\begin{equation*}
		\forall x \forall y \forall z \; components\_of(x,y) \land worn\_by(y,z) \supset targetGender(x,z)
	\end{equation*}
	\begin{Verbatim}[frame=lines,gobble=2,numbers=left]
	 (all x all y all z 
	 (components_of(x,y) & worn_by(y,z))
	 ->
	 (targeted_to(x,z))). 
	\end{Verbatim}
	
	\end{enumerate}

%\subsubsection{Inverse Property}
\clearpage

\subsection{Unsorted or Rejected}
\begin{enumerate}
%\item If an outfit has an garment of clothing that wornOver the legs and an garment of clothing that wornOver the torso and the garments are of the same type then the garments are the same
%\begin{equation*}
%\begin{corollary}
%	\forall(a,b,c,d,x,y,z) \; (garment(a) \land garment(b) \land outfit(c) \land type(d) \land torso(x) \land legs(y) \\
%	\land garmentOf(a,c) \land garmentOf(b,c) \land of\_type(a,d) \land of\_type(b,d) \land wornOver(a,x) \land wornOver(b,y) \\
%	\supset a=b)
%\end{split}
%\end{equation*}

\item If an outfit consist of a type of clothing then there exists an garment of clothing that is a part of that outfit of that same type 
\begin{equation*}
	\forall(x,y,z) \; (type(x) \land outfit(y) \land garmentOf(x,y) \supset (\exists w \; garment(w) \land of\_type(w,x) \land garmentOf(w,y)))
\end{equation*}

\item If an outfit consists on an garment of clothing then the outfit consists of that type of clothing
\begin{equation*}
	\forall(w,x,y,z) \; (garment(w) \land outfit(x) \land garmentOf(x,y) \land type(z) \land of\_type(w,z) \supset garmentOf(z,x))
\end{equation*}

\item No type of clothing can be a type of itself
\begin{equation*}
	\forall(x,y) \; (type(x) \land type(y) \supset \lnot(x=y))
\end{equation*}

\item No outfit can consist of the same two garments of clothing
\begin{equation*}
	\forall(x,y,z) \; (garment(x) \land garment (y) \land outfit(z) \land garmentOf(x,z) \land garmentOf(y,z) \supset \lnot(x=y))
\end{equation*}

\item garments of clothing that are of the same type and part of the same outfit are the same garment of clothing 
\begin{equation*}
	\begin{split}
	\forall(t,x,y,z) \; (garment(x) \land garment (y) \land outfit(z) \land type(t) \\
	\land garmentOf(x,z) \land garmentOf(y,z) \land of\_type(x,t) \land of\_type(y,t) \supset (x=y))
	\end{split}
\end{equation*}

\item No outfit can consist of the same two types of clothing
\begin{equation*}
	\forall(x,y,z) \; (type(x) \land type(y) \land outfit(z) \land garmentOf(x,z) \land garmentOf(y,z) \supset \lnot(x=y))
\end{equation*}

%\item All garments of a subtype of clothing are also an garment of clothing of the subtypes type
%\begin{equation*}
%	\forall(x,y,z) \; (garment(x) \land type(y) \land type(z) \land of\_type(x,y) \supset \land of\_type(x,z))
%\end{equation*}

\item All garments that cover feet are of type shoes or of type socks
\begin{equation*}
	\begin{split}
	\forall(x,y,z) \; (garment(x) \land feet(y) \land cover(x,y) \supset (shoes(z) \land of\_type(x,z)) \\ \lor (socks(z) \land of\_type(x,z)))
	\end{split}
\end{equation*}

%\item All outfits with socks must have shoes
%\begin{equation*}
%	\forall(x,y) \; (socks(x) \land outfit(y) \land garmentOf(x,y) \supset (\exists z \; shoes(z) \land garmentOf(x,y)))
%\end{equation*}


\item Dresses cover the legs and torso
\begin{equation*}
	\forall(x,y,z) \; (dress(x) \land torso(y) \land legs(z) \supset wornOver(x,y) \land wornOver(x,z))
\end{equation*}

\item An garment of clothing is appropriate for only one gender
\begin{equation*}
	\begin{split}
	\forall(x,y,z) \; (garment(x) \land male(y) \land female(z) \land gender\_appr(x,y) \\ \supset \lnot gender\_appr(x,z))
	\end{split}
\end{equation*}

\item All garments of clothing of type dress are appropriate for women.
\begin{equation*}
	\forall(x,y,z) \; (garment(x) \land dress(y) \land of\_type(x,y) \land female(z) \supset gender\_appr(x,z)) 
\end{equation*}

\item All garments of clothing of type skirt are appropriate for women.
\begin{equation*}
	\forall(x,y,z) \; (garment(x) \land skirt(y) \land of\_type(x,y) \land female(z) \supset gender\_appr(x,z)) 
\end{equation*}

\item All garments of clothing of type women's shoes are appropriate for women.
\begin{equation*}
	\forall(x,y,z) \; (garment(x) \land womens\_shoes(y) \land of\_type(x,y) \land female(z) \supset gender\_appr(x,z)) 
\end{equation*}

\item All garments of clothing of type men's shoes are appropriate for men.
\begin{equation*}
	\forall(x,y,z) \; (garment(x) \land mens\_shoes(y) \land of\_type(x,y) \land men(z) \supset gender\_appr(x,z)) 
\end{equation*}

\item All garments of clothing of type blouse are appropriate for women.
\begin{equation*}
	\forall(x,y,z) \; (garment(x) \land blouse(y) \land of\_type(x,y) \land women(z) \supset gender\_appr(x,z)) 
\end{equation*}

%\item Women's shoes are an garment of clothes of type shoes for females
%\begin{equation*}
%	\forall(w,x,y,z) \; (womens\_shoe(w) \supset garment(x) \land shoes(y) \land of\_type(x,y) \land female(z) \land gender\_appr(x,z))  
%\end{equation*}

\item All combinations of skirts, women's jackets and blouses make a skirt suit.
\begin{equation*}
	\forall(w,x,y,z) skirtSuit(w) \supset skirt(x) \land womens\_jacket(y) \land blouse(z) 
\end{equation*}

\item All combinations of dresses and women's jackets make a dress suit
\begin{equation*}
	\forall(x,y,z) skirtSuit(x) \supset dress(y) \land womens\_jacket(z) 
\end{equation*}

\end{enumerate}

\end{document}